\renewcommand{\baselinestretch}{1.5} %設定行距
\pagenumbering{roman} %設定頁數為羅馬數字
\clearpage  %設定頁數開始編譯
\sectionef
\addcontentsline{toc}{chapter}{摘~~~要} %將摘要加入目錄
\begin{center}
\LARGE\textbf{摘~~~要}\\
\end{center}

%\begin{flushleft}
\raggedright
\fontsize{14pt}{20pt}\sectionef\hspace{12pt}\quad 本研究旨在探討如何利用ODOO PLM進行協同設計,以提高團隊合作效率和品質。通過分析ODOO PLM在協同設計之過程中的應用效果,並提出相關的優化建議,以改善設計流程並推動協同設計的應用。\\[14pt]

\fontsize{14pt}{20pt}\sectionef\hspace{12pt}\quad 以鋼球平衡台設計為例,我們將透過ODOO PLM和GitHub進行協同設計、管理、製造執行及整合功能。設計的過程中,我們將會使用Geogebra、Onshape和Solidworks等工具設計機構,並透過CoppeliaSim和Python進行PID控制模擬。同時,使用自行維護的3D列印機製作所需零件,以實現虛實整合之目標。最後根據ODOO PLM和GitHub的記錄歷程,評估協同作業的工作模式。\\

%\end{flushleft}



\vspace{6cm}


\begin{hangparas}{1.5cm}{1}[b!]
\fontsize{14pt}{20pt}\sectionef
關鍵字:比例-積分-微分控制器 (PID)、產品生命週期管理 (PLM)、協同 (CD)、CoppeliaSim、Github
\end{hangparas}
\newpage
%=--------------------Abstract----------------------=%
\begin{center}
\renewcommand{\baselinestretch}{1.5} %設定行距
\addcontentsline{toc}{chapter}{Abstract} %將摘要加入目錄
\LARGE\textbf\sectionef{Abstract}\\
\end{center}

\fontsize{14pt}{16pt}\sectionef\hspace{12pt}\quad This study aims to explore the utilization of ODOO PLM for collaborative design to enhance team cooperation efficiency and quality. By analyzing the application effectiveness of ODOO PLM in collaborative design processes and proposing relevant optimization suggestions, the research seeks to improve design workflows and promote the application of collaborative design.\\[12pt]

\fontsize{14pt}{16pt}\sectionef\hspace{12pt}\quad Using the design of a steel ball balancing platform as an example, collaborative design, management, manufacturing execution, and integration functionalities will be conducted through ODOO PLM and GitHub. Throughout the design process, tools such as Geogebra, Onshape, and Solidworks will be employed to design mechanisms, with CoppeliaSim and Python utilized for PID control simulation. Additionally, required components will be fabricated using a self-maintained 3D printer to achieve the goal of virtual and real integration. Finally, based on the record history of ODOO PLM and GitHub, the collaborative operation mode will be evaluated.\\



\vspace{3cm}
% \fontsize{14pt}{16pt}\selectfont\sectionef Keywords: proportional–integral–derivative controller (PID), Product Lifecycle Management (PLM),collaborative(CD), CoppeliaSim ,Github
\begin{hangparas}{1.5cm}{1}
Keywords: proportional–integral–derivative controller (PID), Product Lifecycle Management (PLM),collaborative(CD), CoppeliaSim ,Github
\end{hangparas}



