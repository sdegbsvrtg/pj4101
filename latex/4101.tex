\documentclass[14pt,a4paper]{report}  %紙張設定
\usepackage{xeCJK}%中文字體模組
%
% Linux 字型設定
\setCJKmainfont{MoeStandardKai.ttf}% Linux 系統中文設定
\newfontfamily\sectionef{Nimbus Roman}%Linux 系統英文設定
%
% Windows 字型設定
%\setCJKmainfont{標楷體} %設定中文字體
%\newfontfamily\sectionef{Times New Roman}%設定英文字體
%
\usepackage{enumerate}
\usepackage{amsmath,amssymb}%數學公式、符號
\usepackage{amsfonts} %數學簍空的英文字
\usepackage{graphicx, subfigure}%圖形
\usepackage{fontawesome5} %引用icon
\usepackage{type1cm} %調整字體絕對大小
\usepackage{textpos} %設定文字絕對位置
\usepackage[top=2.5truecm,bottom=2.5truecm,
left=3truecm,right=2.5truecm]{geometry}
\usepackage{titlesec} %目錄標題設定模組
\usepackage{titletoc} %目錄內容設定模組
\usepackage{textcomp} %表格設定模組
\usepackage{multirow} %合併行
%\usepackage{multicol} %合併欄
\usepackage{CJK} %中文模組
\usepackage{CJKnumb} %中文數字模組
\usepackage{wallpaper} %浮水印
\usepackage{listings} %引用程式碼
\usepackage{hyperref} %引用url連結
\usepackage{setspace}
\usepackage{lscape}%設定橫式
\lstset{language=Python, %設定語言
		basicstyle=\fontsize{14pt}{4pt}\selectfont, %設定程式內文字體大小
		frame=lines,	%設定程式框架為線
}
%\usepackage{subcaption}%副圖標
\graphicspath{{./../images/}} %圖片預設讀取路徑
\usepackage{indentfirst} %設定開頭縮排模組
\renewcommand{\figurename}{\Large 圖 } %更改圖片標題名稱
\renewcommand{\tablename}{\Large 表 }
\renewcommand{\lstlistingname}{\Large 程式 } %設定程式標示名稱
\hoffset=-5mm %調整左右邊界
\voffset=-8mm %調整上下邊界
\setlength{\parindent}{3em}%設定首行行距縮排
\usepackage{appendix} %附錄
\usepackage{diagbox}%引用表格
\usepackage{multirow}%表格置中
%\usepackage{number line}
\usepackage{hanging}
%=------------------更改標題內容----------------------=%
\titleformat{\chapter}[hang]{\center\sectionef\fontsize{20pt}{1pt}\bfseries}{\LARGE 第\CJKnumber{\thechapter}章}{1em}{}[]
\titleformat{\section}[hang]{\sectionef\fontsize{18pt}{2.5pt}\bfseries}{{\thesection}}{0.5em}{}[]
\titleformat{\subsection}[hang]{\sectionef\fontsize{16pt}{2.5pt}\bfseries}{{\thesubsection}}{1em}{}[]
\titleformat{\subsubsection}[hang]{\sectionef\fontsize{14pt}{2.5pt}}{{\thesubsubsection}}{1em}{}[]
%=------------------更改目錄內容-----------------------=%
\titlecontents{chapter}[11mm]{}{\sectionef\fontsize{18pt}{2.5pt}\bfseries\makebox[3.5em][l]
{第\CJKnumber{\thecontentslabel}章}}{}{\titlerule*[0.7pc]{.}\contentspage}
\titlecontents{section}[18mm]{}{\sectionef\LARGE\makebox[1.5em][l]
{\thecontentslabel}}{}{\titlerule*[0.7pc]{.}\contentspage}
\titlecontents{subsection}[4em]{}{\sectionef\Large\makebox[2.5em][l]{{\thecontentslabel}}}{}{\titlerule*[0.7pc]{.}\contentspage}
\titlecontents{subsubsection}[4em]{}{\sectionef\Large\makebox[2.5em][l]{{\thecontentslabel}}}{}{\titlerule*[0.7pc]{.}\contentspage}
%=----------------------章節間距----------------------=%
\titlespacing*{\chapter} {0pt}{0pt}{18pt}
\titlespacing*{\section} {0pt}{12pt}{6pt}
\titlespacing*{\subsection} {0pt}{6pt}{6pt}
\titlespacing*{\subsubsection} {0pt}{6pt}{6pt}
%=----------------------標題-------------------------=%             
\begin{document} %文件
\sectionef %設定英文字體啟用
\vspace{12em}
\begin{titlepage}%開頭
\begin{center}   %標題  
\makebox[1.5\width][s]
{\fontsize{24pt}{2.5pt}國立虎尾科技大學}\\[18pt]
\makebox[1.2\width][s]
{\fontsize{24pt}{2.5pt}機械設計工程系暨精密機械工程科}\\[18pt]
\makebox[1.5\width][s]
{\fontsize{24pt}{2.5pt}專題製作報告}\\[18pt]
%設定文字盒子 [方框寬度的1.5倍寬][對其方式為文字平均分分布於方框中]\\距離下方18pt
\vspace{6em} %下移

\fontsize{30pt}{1em}\selectfont\textbf{\vspace{0.5em}ODOO PLM 在協同設計上的應用 - 以鋼球平衡台設計為例}\\

\vspace{1em}
\sectionef\fontsize{30pt}{1em}\selectfont\textbf
{Application of ODOO PLM \\in collaborative design \\- taking the Design of Steel Ball Balancing Platform as an example}
 \vspace{1em}
%=---------------------參與人員-----------------------=%             
\end{center}
\begin{flushleft}
\begin{LARGE}

\hspace{32mm}\makebox[5cm][s]
{指導教授:\quad 嚴\quad 家\quad 銘\quad 老\quad 師}\\[6pt]


\hspace{32mm}\makebox[5cm][s]
{班\qquad 級:\quad 四\quad 設\quad 三\quad 乙}\\[6pt]
\hspace{32mm}\makebox[5cm][s]
{學\qquad 生:\quad 陳\quad 岳\quad 檉\quad(41023218)}
\\[6pt]
\hspace{32mm}\makebox[5cm][s]
{\hspace{36.5mm}蔡\quad 弦\quad 霖\quad(41023248)}\\[6pt]
\hspace{32mm}\makebox[5cm][s]
{\hspace{36.5mm}鄭\quad 立\quad 揚\quad(41023251)}\\[6pt]
\hspace{32mm}\makebox[5cm][s]
{\hspace{36.5mm}謝\quad 鴻\quad 元\quad(41023254)}\\[6pt]
%設定文字盒子[寬度為5cm][對其方式為文字平均分分布於方框中]空白距離{36.5mm}\空白1em
\end{LARGE}
\end{flushleft}
\vspace{4em}
\fontsize{18pt}{2pt}\selectfont\hspace{1em}\centerline{\makebox[\width][s]
{中華民國\hspace{3em} 
\hspace*{-1em}一~一~三\quad 年\quad 六\quad 月}}
\end{titlepage}
\newpage
%=---------------專題製作合可證明---------------------=%
 {\renewcommand\baselinestretch{1.4}\selectfont %設定以下行距
 {\begin{center}
    {\fontsize{20pt}{2.5pt} {國立虎尾科技大學}\\[8pt]{機械設計工程系暨精密機械工程科}\\[8pt]{學生專題製作合格認可證明}\\
    \hspace*{\fill} \\ %似enter鍵換行
    \par}
     \end{center}}
    {\begin{textblock}{60}(1.85,0.8)
    \noindent \fontsize{15pt}{16pt}\selectfont 專題製作修習學生\enspace:\quad
    {\begin{minipage}[t]{10em}\underline{四設三乙\enspace 41023218\enspace 陳岳檉}\\ \underline{四設三乙\enspace 41023248\enspace 蔡弦霖}\\ \underline{四設三乙\enspace 41023251\enspace 鄭立揚}\\ \underline{四設三乙\enspace 41023254\enspace 謝鴻元}\\ %下劃線符號指令
    \end{minipage}}
         \par} %結束指定行距
    {\renewcommand\baselinestretch{1.2}\selectfont %設定以下行距
    {\begin{textblock}{30}(1.8,4)
    \noindent \fontsize{16pt}{16pt}\selectfont 專題製作題目\enspace :ODOO PLM 在協同設計上的應用\\-以鋼球平衡台設計為例
    \hspace*{\fill} \\
    \hspace*{\fill} \\
    \noindent \fontsize{16pt}{16pt}\selectfont 經評量合格,特此證明
    \hspace*{\fill} \\
    \hspace*{\fill} \\
    \noindent \fontsize{16pt}{16pt} \makebox[6em][s]{評審委員}\enspace:\quad
    {\begin{minipage}[t]{6em} \underline{            }\\[16pt] \underline{            }\\[16pt] \underline{            }\\
    \end{minipage}}
    \end{textblock}}
    {\begin{textblock}{10}(1.8,9)
    {\begin{flushleft}
    \fontsize{16pt}{16pt}\selectfont \makebox[6em][s]{指導老師}\enspace:\quad \underline{            }\\[10pt]
    \fontsize{16pt}{16pt}\selectfont \makebox[6em][s]{系主任}\enspace:\quad \underline{            }\\
    \hspace*{\fill} \\
    \fontsize{16pt}{2.5pt}\selectfont \makebox[12em][s]{中華民國~一一三年}\hspace{2pt}
    \fontsize{16pt}{2.5pt}\selectfont\makebox[8em][s]{6月5日}
    \end{flushleft}}
    \end{textblock}}
    \end{textblock}}
     \par} %結束指定行距
    \thispagestyle{empty}
     \newpage
%=------------------------摘要-----------------------=%
\renewcommand{\baselinestretch}{1.5} %設定行距
\pagenumbering{roman} %設定頁數為羅馬數字
\clearpage  %設定頁數開始編譯
\sectionef
\addcontentsline{toc}{chapter}{摘~~~要} %將摘要加入目錄
\begin{center}
\LARGE\textbf{摘~~~要}\\
\end{center}

%\begin{flushleft}
\raggedright
\fontsize{14pt}{20pt}\sectionef\hspace{12pt}\quad 本研究旨在探討如何利用ODOO PLM進行協同設計,以提高團隊合作效率和品質。通過分析ODOO PLM在協同設計之過程中的應用效果,並提出相關的優化建議,以改善設計流程並推動協同設計的應用。\\[14pt]

\fontsize{14pt}{20pt}\sectionef\hspace{12pt}\quad 以鋼球平衡台設計為例,我們將透過ODOO PLM和GitHub進行協同設計、管理、製造執行及整合功能。設計的過程中,我們將會使用Geogebra、Onshape和Solidworks等工具設計機構,並透過CoppeliaSim和Python進行PID控制模擬。同時,使用自行維護的3D列印機製作所需零件,以實現虛實整合之目標。最後根據ODOO PLM和GitHub的記錄歷程,評估協同作業的工作模式。\\

%\end{flushleft}



\vspace{11cm}


\begin{hangparas}{1.5cm}{1}
\fontsize{14pt}{20pt}\sectionef
關鍵字:比例-積分-微分控制器 (PID)、產品生命週期管理 (PLM)、協同 (CD)、CoppeliaSim、Github
\end{hangparas}
\newpage
%=--------------------Abstract----------------------=%
\begin{center}
\renewcommand{\baselinestretch}{1.5} %設定行距
\addcontentsline{toc}{chapter}{Abstract} %將摘要加入目錄
\LARGE\textbf\sectionef{Abstract}\\
\end{center}

\justifying\fontsize{14pt}{16pt}\sectionef\hspace{12pt}\quad This study aims to explore the utilization of ODOO PLM for collaborative design to enhance team cooperation efficiency and quality. By analyzing the application effectiveness of ODOO PLM in collaborative design processes and proposing relevant optimization suggestions, the research seeks to improve design workflows and promote the application of collaborative design.\\[12pt]

\justifying\fontsize{14pt}{16pt}\sectionef\hspace{12pt}\quad Using the design of a steel ball balancing platform as an example, collaborative design, management, manufacturing execution, and integration functionalities will be conducted through ODOO PLM and GitHub. Throughout the design process, tools such as Geogebra, Onshape, and Solidworks will be employed to design mechanisms, with CoppeliaSim and Python utilized for PID control simulation. Additionally, required components will be fabricated using a self-maintained 3D printer to achieve the goal of virtual and real integration. Finally, based on the record history of ODOO PLM and GitHub, the collaborative operation mode will be evaluated.\\



\vspace{9cm}
% \fontsize{14pt}{16pt}\selectfont\sectionef Keywords: proportional–integral–derivative controller (PID), Product Lifecycle Management (PLM),collaborative(CD), CoppeliaSim ,Github
\begin{hangparas}{1.5cm}{1}
Keywords: proportional–integral–derivative controller (PID), Product Lifecycle Management (PLM),collaborative(CD), CoppeliaSim ,Github
\end{hangparas}




\newpage

%=------------------------謝辭----------------------=%
\begin{center}
\addcontentsline{toc}{chapter}{誌~~~謝}
\centerline\LARGE\textbf{誌~~謝}\\
\end{center}
\begin{flushleft}
\fontsize{14pt}{2.5pt}\hspace{12pt}\quad 本專題能完成有著許多人員的幫忙,大四學長他們不吝嗇地將往年的製作經驗傳授給我們,讓我們在製作的時候少走了許多錯路,還總是貼心找出重點提醒我們可以加以描述。再來是我們的指導教授嚴家銘教授,他提供了多方面的資訊,拋出問題並給予建議,擬定了我們小組研究和學習的方向,討論也時常提出建議以及未來發展,得以順利解決遇到的技術問題,同時也給了相當程度的自由,讓小組得以有彈性去尋探索及摸索,而本專題組員也充分地付出了許多,讓專題研究能順利完成,從中獲益良多,特此感謝。 
\end{flushleft}
\newpage
%=------------------------目錄----------------------=%

\addcontentsline{toc}{chapter}{目錄}
\renewcommand{\contentsname}{\centerline{\fontsize{18pt}{\baselineskip}\selectfont\textbf{目\quad 錄}}}
\tableofcontents  %目錄產生
\newpage
%=------------------圖表目錄產生----------------------=%
\renewcommand{\listfigurename}{\centerline{\fontsize{18pt}{\baselineskip}\selectfont\textbf{圖\quad 目\quad 錄 }}}
\newcommand{\loflabel}{圖} %定義\loflabel 文字為圖
\renewcommand{\numberline}[1]{\loflabel~\fontsize{14pt}{12pt}\selectfont #1\hspace*{0.5em}}
\listoffigures

\newpage

\renewcommand{\listtablename}{\centerline{\fontsize{18pt}{\baselineskip}\selectfont\textbf{表\quad 目\quad 錄 }}}
\newcommand{\lotlabel}{表} %定義\lotlabel 文字為表
\renewcommand{\numberline}[1]{\lotlabel~\fontsize{14pt}{12pt}\selectfont #1\hspace*{0.5em}}
\listoftables
%----------------------------------------%


%=-------------------------內容----------------------=%

%\input{1_introduction.tex}
%\input{2_FEM.tex}

\input{01_introduction.tex}
\input{02_Background and Motion.tex}
\input{03_CH3.tex}
\input{04_CH4.tex}
\chapter{ODOO PLM 在協同設計中的應用}
在這章節中我們將使用1D系統的鋼球平衡台來展示ODOO中產品生命週期(PLM)的功能,首先我們來到ODOO主畫面(圖\ref{ODOO主畫面})並選取產品生命週期(PLM),進到PLM概覽(圖\ref{PLM概覽})之後選擇主資料選單中的產品。


\begin{figure}[h!]
\centering
\includegraphics[width=1\textwidth]{../images/圖5.1.png}
\caption{ODOO主畫面}
\label{ODOO主畫面}
\end{figure}
\begin{figure}[h!]
\centering
\includegraphics[width=1\textwidth]{../images/圖5.2.png}
\caption{PLM概覽}\label{PLM概覽}
\end{figure}
接下來按下新增產品(圖\ref{新增產品})進入到新增畫面(圖\ref{新增畫面}),我們以鋼球平衡台(圖\ref{鋼球平衡台})作為範例。選擇物料清單(圖\ref{物料清單})來新增鋼球平衡台所需的零件。\\

\begin{figure}[h!]
\centering
\includegraphics[width=0.4\textwidth]{../images/圖5.3.png}
\caption{新增產品}\label{新增產品}
\end{figure}

\begin{figure}[h!]
\centering
\includegraphics[width=0.8\textwidth]{../images/圖5.4.png}
\caption{新增畫面}\label{新增畫面}
\end{figure}

\begin{figure}[h!]
\centering
\includegraphics[width=0.8\textwidth]{../images/圖5.5.png}
\caption{鋼球平衡台}\label{鋼球平衡台}
\end{figure}

\begin{figure}[h!]
\centering
\includegraphics[width=1\textwidth]{../images/圖5.6.png}
\caption{物料清單}\label{物料清單}
\end{figure}


將所需零件加入後,這些零組件會自動出現在剛才提到的產品中(圖\ref{產品列表}),將產品設定完後回到PLM概覽並選取新產品介紹下方的工程變更(圖\ref{工程變更選項})。\\

\begin{figure}[h!]
\centering
\includegraphics[width=1\textwidth]{../images/圖5.7.png}
\caption{產品列表}\label{產品列表}
\end{figure}

\begin{figure}[h!]
\centering
\includegraphics[width=1\textwidth]{../images/圖5.8.png}
\caption{工程變更選項}\label{工程變更選項}
\end{figure}

在這個工程變更指令頁面(圖\ref{工程變更指令頁面})中我們選擇要製作的產品和物料清單,並且指派工作給各單位組員也可以設定完成期限或留下備註。\\

\begin{figure}[h!]
\centering
\includegraphics[width=1\textwidth]{../images/圖5.9.png}
\caption{工程變更指令頁面}\label{工程變更指令頁面}
\end{figure}

當我們設定完成後製作鋼球平衡台這項任務就會出現在(圖\ref{工程變更指令總覽})頁面上。團隊中的主管可以藉由拖曳將圖塊任務移到相對應的狀態底下,假如專案已完成,主管可將圖塊移到已完成區域(圖\ref{已完成之項目}),這些狀態可依情形不同做修改或增加。\\


\begin{figure}[h!]
\centering
\includegraphics[width=1\textwidth]{../images/圖5.10.png}
\caption{工程變更指令總覽}\label{工程變更指令總覽}
\end{figure}


\begin{figure}[h!]
\centering
\includegraphics[width=1\textwidth]{../images/圖5.11.png}
\caption{已完成之項目}\label{已完成之項目}
\end{figure}

若想更改用料清單可以使用主頁面中用料清單(BOM)更新的功能,使用方法與建立新產品雷同此處就不多贅述。

\newpage
\chapter{案例研究:鋼球平衡台的設計}
在鋼球平衡台中我們會用到兩種不同領域的理論,數學系統模型以牛頓力學推導運動方程式後使用拉氏轉換將時域轉變成頻域,而另外一項就是自動控制中常見的PID控制器。\\

\section{數學系統模型}
球體的動態是由物理定律推導出,以微分方程式來表達,我們將使用牛頓力學來得到球的運動方程式,並使用拉氏轉換解之。\\

\subsection{簡化與假設}
為了得到球在平板上的運動方程式我們需假設球的幾何型態是完全球形且均質、球在平台上只在X方向移動、球在平台上只做滾動無滑動並且不考慮摩擦力。\\

\subsection{運動方程式}
球的絕對加速度方程式由參考書籍[7]得到。\\

\begin{equation}
\mathbf{a}_a=\dot{\omega} \times \mathbf{r}+\omega \times(\omega \times \mathbf{r})+2 \omega \times \mathbf{v}_{\text {rel }}+\mathbf{a}_{\text {rel }}
\label{((6.1))}
\end{equation} \\

接下來我們將6.1式改寫為\ref{(6.2)},式中\(\mathbf{e}_{k1}\)和\(\mathbf{e}_{i1}\)代表單位向量,\(x_p\)代表球相對於座標系的位置,\(\alpha_1\)代表平台的傾角。

\begin{equation}
\mathbf{a}_1=\ddot{\alpha_1} \mathbf{e}_{k1} \times x_p \mathbf{e}_{i1}+\dot{\alpha_1} \mathbf{e}_{k1} \times\left(\dot{\alpha_1} \mathbf{e}_{k1} \times x_p \mathbf{e}_{i1}\right)+2 \dot{\alpha_1} \mathbf{e}_{k1} \times \dot{x_p} \mathbf{e}_{i1}+\ddot{x_p} \mathbf{e}_{i1}
\label{(6.2)}
\end{equation} \\

將\ref{(6.2)}經過簡化整理後得到\\

\begin{equation}
\mathbf{a}_1= \left( \ddot{x_p} - x_p \dot{\alpha_1}^2 \right) \mathbf{e}_{i1} + \left( x_p \ddot{\alpha_1} + 2 \dot{\alpha_1} \dot{x_p} \right) \mathbf{e}_{j1}
\label{(6.3)}
\end{equation}\\

\begin{figure}[h]
\centering
\includegraphics[width=0.7\textwidth]{../images/圖6-1.png}
\caption{平台角度和球體位置的關係圖}
\label{fig:6-1}
\end{figure}

在圖\ref{fig:6-1}的自由體圖中,從力矩的平衡可以看出球的剩餘力。\\

\begin{equation}
I_b \ddot{\beta}_1=F_{r1} r
\label{(6.4)}
\end{equation} \\

\(I_b\)是球的質量慣性矩,\(\beta_1\)是球相對於其初始位置在平台中心的角度,\(r\)是球的半徑,\(F_r\)是來自平台對球的作用力,我們假設求在平台上並無滑動所以我們可以根據位置定義相對角度\(\beta\)。\\

\begin{equation}
\beta_1=-\frac{x_p}{r}
\label{(6.5)}
\end{equation} \\

為了求解\ref{(6.4)}中的\(F_r\),我們將\ref{(6.5)}式的二次時間導數代入\ref{(6.4)}式中得到\ref{(6.6)}式。\\

\begin{equation}
F_r=-\frac{I_b \ddot{x_p}}{r^2}
\label{(6.6)}
\end{equation} \\

球在平台上受到的力和平台對球施加的力之間的平衡,由\ref{(6.3)}式中的加速度和\ref{(6.6)}中的力導致,由此得到動態系統的運動方程式。\\

\begin{equation}
\left(\frac{I_b}{r^2}+m_b\right) \ddot{x_p}+m_b g \sin \alpha_1-m_b x_p{\dot{\alpha_1}}^2=0
\label{(6.7)}
\end{equation} \\

為了做拉式轉換我們稍微整理方程式。\\

\begin{equation}
\ddot{x}=\frac{m_b r_b^2\left(x_p \dot{\alpha}_1^2-g \sin \alpha_1\right)}{m_b r_b^2+I_b}
\label{(6.8)}
\end{equation} \\

接下來我們在\(X_p=0\),\(\alpha_1=0\)對\ref{(6.8)}式作線性化。\\

\begin{equation}
\ddot{x}=\frac{m_b g \alpha_1 r^2}{m_b r_b{ }^2+I_b}
\label{(6.9)}
\end{equation}\\

 \newpage


當\(\alpha_1\)出現小變動時線性化可得\ref{(6.9)}式。接下來當我們將\(I_b\)也就是球體的質量慣性矩代入我們可以得到\ref{(6.10)},我們可以觀察到該系統的運動方程式和該球體的半徑和質量無關。\\

\begin{equation}
\ddot{x}=\frac{5}{7} g \alpha_1
\label{(6.10)}
\end{equation}\\

最後我們對\ref{(6.10)}作拉式轉換得到\ref{(6.11)}。\\

\begin{equation}
s^2 X=\frac{5}{7} g A_1
\label{(6.11)}
\end{equation}\\

在推導運動運動方程式後我們得到了平台角度和球的關係式,
接下來我們利用geogebra進行模擬得到馬達轉角和平台角度的關係圖(圖\ref{(圖 6-2)})。\\

\begin{figure}[h]
\centering
\includegraphics[width=0.7\textwidth]{../images/圖6-2.png}
\caption{馬達轉角和平台角度關係圖}
\label{(圖 6-2)}
\end{figure}



由(圖\ref{(圖 6-2)})可觀察到,超過正70度的部分由於桿件設計反而造成平台角度下降,故將該段行程捨棄,
利用python生成擬合曲線(圖\ref{圖6-3}),得到馬達轉角和平台的關係式\ref{(6.12)}。\\


\begin{figure}[h]
\centering
\includegraphics[width=1\textwidth]{../images/圖6-3.png}
\caption{擬合曲線}
\label{圖6-3}
\end{figure}

\begin{equation}
\alpha=0.233371\theta-0.293753
\label{(6.12)}
\end{equation}\\









\newpage
\section{一維系統架構}

\subsection{設計理念}

我們以鋼球平衡台作為專題的主體,然後寫程式驅動雷射測距感測器當鋼球遠離時platform,當鋼球靠近時platform放下,重複此動作直至鋼球平衡台平衡。

\begin{figure}[h!]
    \centering
    \begin{minipage}[b]{0.45\textwidth}
        \centering
        \includegraphics[width=\textwidth,height=0.22\textheight]{./../images/螢幕擷取畫面 2024-05-22 181158.png}
        \caption{2D草圖(1)}
    \end{minipage}
    \hfill
    \begin{minipage}[b]{0.45\textwidth}
        \centering
        \includegraphics[width=\textwidth,height=0.22\textheight]{./../images/螢幕擷取畫面 2024-05-22 180925.png} 
        \caption{2D草圖(2)}
    \end{minipage}
\end{figure}

\subsection{馬達角度所對應之平台角度關係}

\begin{figure}[htbp]
    \centering
    \includegraphics[width=1\textwidth]{./../images/6-1-50}
    \caption{馬達角度關係圖}
\end{figure}

\newpage

使用 SolidWorks 2023 進行繪圖。
\begin{figure}[htbp]
    \centering
    \includegraphics[width=0.5\textwidth]{./../images/6-1-1}
    \caption{SOLIDWORKS}
\end{figure}

\subsection{platform}

第一版本鋼球平衡台的 platform 軌道長度為 200mm 整體的寬度為 30mm 並給定深度填料 11mm。

\begin{figure}[h!]
    \centering
    \begin{minipage}[b]{0.6\textwidth}
        \centering
        \includegraphics[width=\textwidth,height=0.22\textheight]{./../images/6-1-11}
        \caption{PLATFORM草圖 (1)}
    \end{minipage}
    \hfill
    \begin{minipage}[b]{0.35\textwidth}
        \centering
        \includegraphics[width=\textwidth,height=0.25\textheight]{./../images/6-1-12} 
        \caption{編輯特徵(1)}
    \end{minipage}
\end{figure}
    
軌道上方寬度為 8.5mm,下方為 7.2mm,深 4mm並將紅圈處深伸長除料選擇完全貫穿。

\begin{figure}[h!]
    \centering
    \begin{minipage}[b]{0.6\textwidth}
        \centering
        \includegraphics[width=\textwidth,height=0.22\textheight]{./../images/6-1-13}
        \caption{PLATFORM草圖 (2)}
    \end{minipage}
    \hfill
    \begin{minipage}[b]{0.35\textwidth}
        \centering
        \includegraphics[width=\textwidth,height=0.25\textheight]{./../images/6-1-14} 
        \caption{編輯特徵(2)}
    \end{minipage}
\end{figure}

\newpage

下方配合處長 30mm 寬 6mm,繪製好圖形草圖。

\begin{figure}[htbp]
    \centering
    \includegraphics[width=0.7\textwidth]{./../images/6-1-15}
    \caption{PLATFORM草圖 (3)}
\end{figure}

給予尺寸後伸長填料 25mm。
\begin{figure}[h!]
    \centering
    \begin{minipage}[b]{0.6\textwidth}
        \centering
        \includegraphics[width=\textwidth,height=0.22\textheight]{./../images/6-1-16}
        \caption{PLATFORM草圖 (4)}
    \end{minipage}
    \hfill
    \begin{minipage}[b]{0.35\textwidth}
        \centering
        \includegraphics[width=\textwidth,height=0.25\textheight]{./../images/6-1-17} 
        \caption{編輯特徵(3)}
    \end{minipage}
\end{figure}

填料完成後再圖形上畫一個 2.9mm 的小孔並進行伸長除料以便與其他零件配合。

 \begin{figure}[htbp]
        \centering
        \includegraphics[width=0.49\textwidth]{./../images/6-1-18}
        \caption{PLATFORM草圖 (5)}
    \end{figure}

\newpage

第一版 platform 完成圖。

\begin{figure}[htbp]
    \centering
    \includegraphics[width=0.78\textwidth]{./../images/6-1-19}
    \caption{PLATFORM零件圖}
\end{figure}

\textbf{修改部分}

軌道上方增加長 26.2mm 寬 2mm 的貫穿凹槽,用於放置感應器。

\begin{figure}[htbp]
    \centering
    \includegraphics[width=0.42\textwidth]{./../images/6-1-20}
    \caption{修改PLATFORM草圖 (1)}
\end{figure}


下方接合處新增 R10 圓角。

\begin{figure}[htbp]
    \centering
    \includegraphics[width=0.4\textwidth]{./../images/6-1-21}
    \caption{修改PLATFORM草圖 (2)}
\end{figure}

\newpage


最終 Platform 零件圖。

\begin{figure}[htbp]
    \centering
    \includegraphics[width=0.9\textwidth]{./../images/6-1-22}
    \caption{最終PLATFORM零件圖}
\end{figure}

\textbf{3D 列印成果}

\begin{figure}[htbp]
    \centering
    \includegraphics[width=0.8\textwidth]{./../images/6-1-26}
    \caption{PLATFORM 3D列印完成圖}
\end{figure}

\newpage

\subsection{base}

第一版 Base 底的長為 237mm 寬為 150mm。

\begin{figure}[htbp]
    \centering
    \includegraphics[width=1\textwidth]{./../images/6-1-27}
    \caption{BASE草圖(1)}
\end{figure}



在底板長 55mm 寬 54mm 處繪製一個 12mm $\times$ 12mm 的方形柱並向上填料 100mm。

\begin{figure}[h!]
    \centering
    \begin{minipage}[b]{0.6\textwidth}
        \centering
        \includegraphics[width=\textwidth,height=0.35\textheight]{./../images/6-1-28}
        \caption{BASE草圖(2)}
    \end{minipage}
    \hfill
    \begin{minipage}[b]{0.35\textwidth}
        \centering
        \includegraphics[width=\textwidth,height=0.35\textheight]{./../images/6-1-29} 
        \caption{編輯特徵(4)}
    \end{minipage}
\end{figure}

\newpage

在方柱上方繪製一個長 30.28mm 寬 20 的長方體然後在長方體上畫直徑 20mm 的半圓最後在圓的中心繪製一個 3.98mm 的小孔最後在長方體上畫一個長 30mm 寬 10.8mm 的小長方體伸長除料選擇完全貫穿方便與上方 platform 配合。

\begin{figure}[h!]
    \centering
    \begin{minipage}[b]{0.45\textwidth}
        \centering
        \includegraphics[width=\textwidth]{./../images/6-1-30}
        \caption{BASE草圖(3)}
    \end{minipage}
    \hfill
    \begin{minipage}[b]{0.45\textwidth}
        \centering
        \includegraphics[width=\textwidth]{./../images/6-1-31} 
        \caption{BASE草圖(4)}
    \end{minipage}
\end{figure}


在距離方柱中心長 129mm 寬 25mm 處繪製一個長 31mm 寬 20mm 向上填料 7mm 的小平台用來定位馬達。

\begin{figure}[h!]
    \centering
    \begin{minipage}[b]{0.6\textwidth}
        \centering
        \includegraphics[width=\textwidth,height=0.30\textheight]{./../images/6-1-33}
        \caption{BASE草圖(5)}
    \end{minipage}
    \hfill
    \begin{minipage}[b]{0.35\textwidth}
        \centering
        \includegraphics[width=\textwidth,height=0.35\textheight]{./../images/6-1-34} 
        \caption{編輯特徵(5)}
    \end{minipage}
\end{figure}

\newpage

並且在兩邊加畫底 15mm 高 45mm 的三角形支撐架防止馬達晃動。

\begin{figure}[htbp]
    \centering
    \includegraphics[width=0.70\textwidth]{./../images/6-1-35}
    \caption{BASE草圖(6)}
\end{figure}

第一版 base 完成圖。

\begin{figure}[htbp]
    \centering
    \includegraphics[width=0.9\textwidth]{./../images/6-1-36}
    \caption{BASE零件圖}
\end{figure}

\newpage

\textbf{修改部分}

底部去除浪費的部分改以長 165.6mm 圓半徑 22.35mm 的直狹槽代替。

\begin{figure}[htbp]
    \centering
    \includegraphics[width=0.85\textwidth]{./../images/6-1-37}
    \caption{修改BASE草圖(1)}
\end{figure}

為了好收納將左方柱子拔除留下一凹槽方便後續配合及螺絲孔。

\begin{figure}[htbp]
    \centering
    \includegraphics[width=0.85\textwidth]{./../images/6-1-38}
    \caption{修改BASE草圖(2)}
\end{figure}

\newpage

最終 base 完成圖。

\begin{figure}[htbp]
    \centering
    \includegraphics[width=0.9\textwidth]{./../images/6-1-39}
    \caption{最終BASE零件圖}
\end{figure}

\textbf{3D 列印成果}

\begin{figure}[htbp]
    \centering
    \includegraphics[width=0.45\textwidth]{./../images/6-1-25}
    \caption{BASE 3D列印完成圖}
\end{figure}

\newpage

\subsection{support}

從底座拔除的部分目的好收納尺寸都沒有改變。
最終 support 完成圖。

\begin{figure}[htbp]
    \centering
    \includegraphics[width=0.90\textwidth]{./../images/6-1-40}
    \caption{最終SUPPORT零件圖}
\end{figure}

\textbf{3D 列印成果}

\begin{figure}[htbp]
    \centering
    \includegraphics[width=0.70\textwidth]{./../images/6-1-24}
    \caption{SUPPORT 3D列印完成圖}
\end{figure}
\newpage
\subsection{link}

上方連結處與 support 尺寸一致皆為長 30.28mm 寬 20 的長方體然後在長方體上畫直徑 20mm 的半圓最後在圓的中心繪製一個 3.98mm 的小孔最後在長方體上畫一個長 30mm 寬 10.8mm 的小長方體伸長除料選擇完全貫穿方便與上方 platform 配合。

\begin{figure}[htbp]
    \centering
    \includegraphics[width=0.5\textwidth]{./../images/6-1-41}
    \caption{LINK草圖(1)}
\end{figure}

下方為 12mm $\times$ 12mm 的方柱並填料 68mm 方便與馬達進行配合。

\begin{figure}[h!]
    \centering
    \begin{minipage}[b]{0.6\textwidth}
        \centering
        \includegraphics[width=\textwidth,height=0.30\textheight]{./../images/6-1-42}
        \caption{LINK草圖(2)}
    \end{minipage}
    \hfill
    \begin{minipage}[b]{0.35\textwidth}
        \centering
        \includegraphics[width=\textwidth,height=0.35\textheight]{./../images/6-1-43} 
        \caption{編輯特徵(6)}
    \end{minipage}
\end{figure}

\newpage

最終 link 完成圖。

\begin{figure}[htbp]
    \centering
    \includegraphics[width=0.85\textwidth]{./../images/6-1-44}
    \caption{最終LINK零件圖}
\end{figure}

\textbf{3D 列印成果}

\begin{figure}[htbp]
    \centering
    \includegraphics[width=0.46\textwidth]{./../images/6-1-23}
    \caption{LINK 3D列印完成圖}
\end{figure}

\newpage

\subsection{crank}

畫一個長36mm寬5mm的crank填料3mm並留兩個3.2mm的圓孔方便鎖上螺栓配合base與link


\begin{figure}[h!]
    \centering
    \begin{minipage}[b]{0.6\textwidth}
        \centering
        \includegraphics[width=\textwidth,height=0.30\textheight]{./../images/螢幕擷取畫面 2024-05-25 220448}
        \caption{crank草圖(1)}
    \end{minipage}
    \hfill
    \begin{minipage}[b]{0.35\textwidth}
        \centering
        \includegraphics[width=\textwidth,height=0.35\textheight]{./../images/螢幕擷取畫面 2024-05-25 221952} 
        \caption{編輯特徵(7)}
    \end{minipage}
\end{figure}

最終crank完成圖

\begin{figure}[htbp]
    \centering
    \includegraphics[width=0.8\textwidth]{./../images/螢幕擷取畫面 2024-05-25 230501}
    \caption{最終crank零件圖}
\end{figure}

\newpage

\subsection{baffle}

繪製一個配合platform,兩兩相組合的檔塊,透過螺絲逼緊的方式夾緊於platform,保留可調整位置、可拆除的特性,以輔助調整控制參數,如此一來當參數錯誤時,球不會直接飛離平台。而上方的拱型則是避免干擾到雷射感測器的訊號。

\begin{figure}[htbp]
    \centering
    \includegraphics[width=0.8\textwidth]{./../images/螢幕擷取畫面 2024-05-25 224335}
    \caption{baffle草圖(1)}
\end{figure}

最終baffle完成圖

\begin{figure}[htbp]
    \centering
    \includegraphics[width=0.8\textwidth]{./../images/螢幕擷取畫面 2024-05-25 230234}
    \caption{最終baffle零件圖}
\end{figure}

\newpage

\subsection{assemble}

組合完成圖。

\begin{figure}[htbp]
    \centering
    \includegraphics[width=1\textwidth]{./../images/螢幕擷取畫面 2024-05-25 223006}
    \caption{ASSEMBLE組合圖(1)}
\end{figure}

\begin{figure}[htbp]
    \centering
    \includegraphics[width=1\textwidth]{./../images/螢幕擷取畫面 2024-05-25 223134}
    \caption{ASSEMBLE組合圖(2)}
\end{figure}

\begin{figure}[htbp]
    \centering
    \includegraphics[width=1\textwidth]{f_assemble}
    \caption{實體組合圖}
\end{figure}



\subsection{驅動方式}

使用金屬齒輪伺服馬達配合程式控制平台,程式放置於 6-3。


\newpage
\section{控制系統設計與結果}
	最終的實體作品,我們選用Arduino UNO 作為開發板,VL53L0作為傳感器輸入,MG 996R伺服馬達進行輸出。

\lstset{
  language=C++,                % 設定語言
  basicstyle=\ttfamily\small,  % 設定字型和字體大小
  keywordstyle=\color{blue},   % 關鍵字顏色
  commentstyle=\color{darkgray},   % 註釋顏色
  stringstyle=\color{red},     % 字串顏色
  numbers=left,                % 行號顯示位置
  numberstyle=\tiny\color{gray}, % 行號字型和顏色
  stepnumber=1,                % 行號的間隔
  numbersep=5pt,               % 行號與程式碼的距離
  showspaces=false,            % 是否顯示空格
  showstringspaces=false,      % 是否顯示字串中的空格
  tabsize=2,                   % Tab 的大小
  breaklines=true,             % 自動換行
  breakatwhitespace=false,     % 只在空白處換行
  frame=single,                % 設定邊框樣式
  captionpos=b,                % 標題位置
  escapeinside={\%*}{*)},      % 可用於在程式碼中插入 LaTeX 命令
  morekeywords={*,...}         % 如果有額外的關鍵字,可以在這裡加入
}


\subsection{控制系統}
Arduino板的控制程式如下:





\subsubsection{導入函式庫}
\begin{lstlisting}[language=C]
#include <Wire.h>
#include <VL53L0X.h>
#include <Servo.h>


\end{lstlisting}

\subsubsection{宣告物件和常數}
\begin{lstlisting}[language=C]
// 宣告物件
VL53L0X sensor;
Servo motor;

// PID 控制常數
const float kp = 38.2;  // 比例增益
const float ki = 5;   // 積分增益
const float kd = 31.6;   // 微分增益
const float initial_delay = 500; // 初始延遲時間 (微秒)


\end{lstlisting}

\subsubsection{設定指數平滑濾波器參數}
\begin{lstlisting}[language=C]
// 指數平滑濾波器參數
const float alpha = 1;  // 平滑因子
float filtered_distance = 0.0;  // 初始化過濾後的距離

\end{lstlisting}

\subsubsection{初始化設定}
\begin{lstlisting}[language=C]

void setup() {
  Serial.begin(115200);
  Wire.begin();
  
  sensor.setTimeout(500);
  if (!sensor.init()) {
    Serial.println("Failed to detect and initialize sensor!");
    while (1);
  }
  sensor.startContinuous();
  
  motor.attach(9);  // 連接馬達到9號引腳
  motor.write(90);
  delay(initial_delay); // 延遲以確保馬達回到中間位置
  
  last_time = micros();  // 初始化時間
}

\end{lstlisting}

\subsubsection{控制馬達}
\begin{lstlisting}[language=C]
float getDistance() {
  float distance = sensor.readRangeContinuousMillimeters(); // 讀取距離,單位毫米
  if (sensor.timeoutOccurred()) {
    Serial.print("Timeout");
    return -1;
  } else {
    distance = constrain(distance, 35, 300); // 將距離值約束在有效範圍內
    return distance;
  }
}

void resetControlVariables() {
  error_sum = 0.0;
  last_error = 0.0;
}

void controlMotor(float target_distance, float dt) {
  float distance = getDistance();
  if (distance > 0) {
    // 更新過濾後的距離
    filtered_distance = alpha * distance + (1 - alpha) * filtered_distance;

    // 檢測碰撞
    if (distance <= 35 || distance >= 130) {
      resetControlVariables();
      Serial.println("Collision detected, resetting control variables.");
      motor.write(90); // 重置馬達位置
      return;
    }

    Serial.print("Target: ");
    Serial.print(target_distance);
    Serial.print(" mm Distance: ");
    Serial.print(filtered_distance);

    // 計算誤差
    float error = target_distance - filtered_distance;
    Serial.print(" Error: ");
    Serial.print(error);

    // 更新誤差總和
    error_sum += error * dt;

    // 防止積分項累積過多
    error_sum = constrain(error_sum, -1000, 1000);

    // 計算微分項
    float derivative = (dt > 0) ? (error - last_error) / dt : 0;

    // 計算控制信號
    float control_signal = (kp * error + ki * error_sum + kd * derivative);

    // 將控制信號映射到有效的角度範圍內
    int angle = constrain(map(control_signal, -10000, 10000, 0, 180), 36 , 144);
    Serial.print(" Control_signal: ");
    Serial.print(control_signal);
    Serial.print(" Angle: ");
    Serial.println(angle);
    motor.write(angle);

    // 更新上一次的誤差
    last_error = error;
  }
}
\end{lstlisting}

\subsubsection{停止程式}
\begin{lstlisting}[language=C]
void loop() {
  if (Serial.available() > 0) {
    char key = Serial.read();
    if (key == 'q') {
      // 停止馬達
      motor.detach();
      while (1);
    }
  }

  unsigned long current_time = micros();
  float dt = (current_time - last_time) / 1000000.0; // 將微秒轉換為秒
  last_time = current_time;

  controlMotor(90, dt);  // 設置目標距離為 90 mm

  delay(50);  // 延遲以匹配時間步長 (dt)
}
  
\end{lstlisting}

\subsection{設計結果}
在最終的版本由於紅外線時常檢測不到體積較小的鋼球,所以我們採用體積更大的乒乓球來代替鋼球使整體系統更加完善,而系統控制的部分我們藉由調整PID參數已盡可能讓系統趨近穩定,但穩定過後還是會出現些微的震盪,我們猜測可能是因為3D列印的公差所導致。




\chapter{評估與結果}

\section{結果呈現}

\subsection{Repository}
由於協同需求,本組採用Github倉儲作為程式碼交流、圖檔存放、網站維護及架設的環境。
組員透過fork倉儲,編輯後再提出pull request進行merge,即可升級倉儲。

\begin{figure}[h]
    \centering
    \includegraphics[width=0.8\textwidth]{repo}
    \caption{本組專題倉儲}
\end{figure}

在倉儲內可查看組員的更新紀錄,並且倉儲內保留著每一版次的資料,可以隨時查看先前的檔案。

\begin{figure}[h]
    \centering
    \includegraphics[width=0.8\textwidth]{7rep_v}
    \caption{倉儲更新紀錄}
\end{figure}

在倉儲內也可透過設定.yml檔,使用LaTeX生成俱有版次的PDF報告。

\begin{figure}[h]
    \centering
    \includegraphics[width=0.8\textwidth]{7pdf_v}
    \caption{PDF報告釋出版次}
\end{figure}

\subsection{Github Page}
本組專題成果網頁係藉由藉由本組指導教授嚴家銘教授所開發的cmsimde子模組進行維護。\\

\begin{figure}[h]
    \centering
    \includegraphics[width=0.8\textwidth]{cmsimde}
    \caption{cmsimde}
\end{figure}

組員只要在可攜環境的近端倉內使用終端機執行cms.bat,就會以python運行cmsimde資料夾中的wsgi.py,
接著就會透過flaskapp.py等檔案啟動近端的動態網頁。
組員即可透過瀏覽器編輯動態網頁,編輯完成後亦可透過編輯器內的選項將動態網頁轉換為靜態,並推送至Github倉儲中,由遠端倉儲生成網頁,
網址為\url{https://mde.tw/pj4101}。\\

\begin{figure}[h]
    \centering
    \includegraphics[width=0.8\textwidth]{mdepj4101}
    \caption{本組專題網頁}
\end{figure}

在此環境之下,本組組員能非常輕鬆的協同對網頁進行維護。\\




\subsection{PID控制}
在經過計算後,我們得到了KP=38.5,KI=5,KD=31.5。但由於系統誤差,導致系統平衡較慢,因此我們進行手動調整得到控制參數為KP=38.2,KI=5,KD=31.6。\\



\section{結果分析}

\subsection{LaTeX-PDF報告}
因為使用latex製作PDF報告,組員們可以同時編輯報告的不同章節,在近端透過可攜環境中的miktex,可以生成並預覽PDF報告,大大提升了協同的效率。也因為LaTeX對數學公式的支援,在製作報告時我們只要輸入指令,
就能產生各種數學公式符號,不需要逐一點選。在我們的專題環境中只要透過github更新.tex檔案,就會自動生成新版本的PDF報告,相當方便。\\

\subsection{實做成果分析}
在製作實體作品時,我們採用熱融堆疊增材生成技術進行製作,由於沒有發現實驗室中庫存的列印線材已經老化,導致我們在列印時產生許多失敗品,走了不少彎路,最後幸虧有綜二館地下室的
智慧學院鼎力相助,讓我們能夠順利完成這項專題研究。

\chapter{結論}
\renewcommand{\baselinestretch}{10.0} %設定行距
%\pagenumbering{arabic} %設定頁號阿拉伯數字


\fontsize{14pt}{2.5pt}\sectionef\hspace{12pt}


在本次的專題中,我們使用Odoo作為產品設計的基底,並且結合SolidWorks 和 Coppeliasim設計零件及模擬,
再加入github進行協同,最後使用3D列印列印零件加以組裝然後配合機電控制實現從零開始生產並且製造出一個產品。在這個專裡我們學到了每種程式不同功用,
並使用各個程式的優缺點加以揉合後產生一加一大於二的功效。\\






有了這些便捷的軟體後我們可以先在SolidWorks 進行零件的初次設計,然後到Coppeliasim 中進行確實的仿真模擬。
在Coppeliasim我們可以模擬產品在不同條件下的行為和性能,並進行虛擬測試及後續的改進及優化。這使我們能夠更好地了解產品的現實狀況及日後所需改進的問題,並更有效地設計和製造產品。\\


我們認為ODOO不僅能定義產品的屬性還能根據產品的類型選定不同的模組進行使用,甚至還可以設置生產的流程,
物料清單及生產的計畫等等,其中最厲害的莫過於產品生命週期管理(PLM)模組,能從需求到設計開發到產品測試到大量生產到產品維護再到產品停產下架完成一整套的產品週期流程,
可謂是從零到有甚至於再到產品的終結。\\


整個專題我們學到了如何獨當一面的設計和如何與其他成員協同分工,
在這個過程中不僅學到了許多新的技能和知識,更體驗到了團隊合作和創新思維的重要性,這些在未來職場上班時都將會寶貴的經驗。\\




\newpage

\renewcommand{\baselinestretch}{0.5} %設定行距


%\input{4_Design_and_Simulation_Environment.tex}
%\input{5_Quadruped_robot_Design_Kinematic_Simulation.tex}
%\input{6_FEM_of_Quadruped_robot.tex}
%\input{7_Generative_Design.tex}
%\input{8_Conclusion.tex}
%\input{9_Future_Directions.tex}
%=---------------------參考文獻----------------------=%
\input{10_reference.tex}
%=---------------附錄-----------------=%
\addcontentsline{toc}{chapter}{附錄} %新增目錄名稱
\input{11_appendix.tex}
\newpage
%=-------------作者簡介-----------------=%
 \addcontentsline{toc}{chapter}{作者簡介}
    \begin{center}
	\fontsize{20pt}{0em}\selectfont \bf{作者簡介}\\
	\end{center}	
	{\begin{textblock}{6}(0,0.5)
	\begin{figure}
	\includegraphics[width=1.25in]{18}
	\end{figure}
	\end{textblock}}
	{\renewcommand\baselinestretch{0.99}\selectfont %設定以下行距
	{\begin{textblock}{15}(3.5,0.7)%{寬度}(以左上角為原點之右移量,下移量)
	\noindent\fontsize{14pt}{0.5em}\selectfont \makebox[4em][s]{姓名}\enspace:\enspace
    \fontsize{14pt}{0.5em}\selectfont \makebox[4em][s]{陳岳檉}\\     \hspace*{\fill} \\
    \fontsize{14pt}{0.5em}\selectfont \makebox[4em][s]{學號}\enspace:\enspace
    \fontsize{14pt}{0.5em}\selectfont \makebox[4em][s]{41023218} \\ %\makebox為文本盒子
    \hspace*{\fill} \\
    \fontsize{14pt}{0.5em}\selectfont \makebox[4em][s]{就讀學校}\enspace:\enspace
    \fontsize{14pt}{0.5em}\selectfont \makebox[9em][s]{國立虎尾科技大學~機械設計工程系}\\
    \hspace*{\fill} \\
    \fontsize{14pt}{0.5em}\selectfont \makebox[4em][s]{經歷}\enspace:\enspace
    \fontsize{14pt}{0.5em}\selectfont \makebox[9em][s]{台中市致用高級中學~機械科}\\
   
    \end{textblock}}}
   % \hspace*{\fill} \\
   \vspace{4em}
	{\begin{textblock}{6}(0,2.3)
	\begin{figure}
	\includegraphics[width=1.15in]{48} 
    \end{figure}
    \end{textblock}}
    {\renewcommand\baselinestretch{0.99}
    \selectfont %設定以下行距
    {\begin{textblock}{15}(3.5,2.5) %{寬度}(以左上角為原點之右移量,下移量)
\noindent\fontsize{14pt}{0.5em}\selectfont \makebox[4em][s]{姓名}\enspace:\enspace
\fontsize{14pt}{0.5em}\selectfont \makebox[4em][s]{蔡弦霖}\\ 
\hspace*{\fill} \\
\fontsize{14pt}{0.5em}\selectfont \makebox[4em][s]{學號}\enspace:\enspace
\noindent\fontsize{14pt}{0.5em}\selectfont \makebox[4em][s]{41023248} \\ 
\hspace*{\fill} \\
\fontsize{14pt}{0.5em}\selectfont \makebox[4em][s]{就讀學校}\enspace:\enspace
\fontsize{14pt}{0.5em}\selectfont \makebox[9em][s]{國立虎尾科技大學~機械設計工程系}\\
\hspace*{\fill} \\
\fontsize{14pt}{0.5em}\selectfont \makebox[4em][s]{經歷}\enspace:\enspace
\fontsize{14pt}{0.5em}\selectfont \makebox[9em][s]{國立台南高級工業職業學校~機械科}\\

    \end{textblock}}}
    %\hspace*{\fill} \\
    \vspace{4em}
    {\begin{textblock}{6}(0,4.1)
    \begin{figure}
        \includegraphics[width=1.15in]{51} %{}內是圖片文件的相對路徑
    \end{figure}
    \end{textblock}}
    {\renewcommand\baselinestretch{0.99}\selectfont %設定以下行距
    {\begin{textblock}{15}(3.5,4.3) %{寬度}(以左上角為原點之右移量,下移量)
\noindent\fontsize{14pt}{0.5em}\selectfont \makebox[4em][s]{姓名}\enspace:\enspace%\noindent指定首行不進行縮排
\fontsize{14pt}{0.5em}\selectfont \makebox[4em][s]{鄭立揚}\\ 
\hspace*{\fill} \\
\noindent\fontsize{14pt}{0.5em}\selectfont \makebox[4em][s]{學號}\enspace:\enspace
\noindent\fontsize{14pt}{0.5em}\selectfont \makebox[4em][s]{41023251} \\ %\makebox為文本盒子
\hspace*{\fill} \\
\noindent\fontsize{14pt}{0.5em}\selectfont \makebox[4em][s]{就讀學校}\enspace:\enspace
\noindent\fontsize{14pt}{0.5em}\selectfont \makebox[9em][s]{國立虎尾科技大學~機械設計工程系}\\
\hspace*{\fill} \\
\noindent\fontsize{14pt}{0.5em}\selectfont \makebox[4em][s]{經歷}\enspace:\enspace
\fontsize{14pt}{0.5em}\selectfont \makebox[9em][s]{台南市黎明高級中學}\\
    \end{textblock}}}
   % \hspace*{\fill} \\
   \vspace{4em}
    {\begin{textblock}{6}(0,5.9)
    \begin{figure}
        \includegraphics[width=1.15in]{54} %{}內是圖片文件的相對路徑
    \end{figure}
    \end{textblock}}
    {\renewcommand\baselinestretch{0.99}\selectfont %設定以下行距
    {\begin{textblock}{15}(3.5,6.1) %{寬度}(以左上角為原點之右移量,下移量)
\noindent\noindent\fontsize{14pt}{0.5em}\selectfont \makebox[4em][s]{姓名}\enspace:\enspace
\noindent\fontsize{14pt}{0.5em}\selectfont \makebox[4em][s]{謝鴻元}\\ \hspace*{\fill} \\
\noindent\fontsize{14pt}{0.5em}\selectfont \makebox[4em][s]{學號}\enspace:\enspace
\noindent\fontsize{14pt}{0.5em}\selectfont \makebox[4em][s]{41023254} \\ \hspace*{\fill} \\
\noindent\fontsize{14pt}{0.5em}\selectfont \makebox[4em][s]{就讀學校}\enspace:\enspace
\noindent\fontsize{14pt}{0.5em}\selectfont \makebox[9em][s]{國立虎尾科技大學~機械設計工程系}\\
\hspace*{\fill} \\
\noindent\fontsize{14pt}{0.5em}\selectfont \makebox[4em][s]{經歷}\enspace:\enspace
\fontsize{12pt}{0.5em}\selectfont \makebox[9em][s]{國立台北科技大學附屬桃園農工高級中等學校~動力科}\\


    \end{textblock}}}
\newpage




%=----------------書背----------------------=%
\pagestyle{empty}%設定沒有頁眉和頁腳
\begin{center}
\fontsize{0.001pt}{1pt}\selectfont .\\
\vspace{2em}
\fontsize{14pt}{30pt}\selectfont 【18】 \\
\fontsize{14pt}{20pt}\selectfont
\vspace{0.1em}
分\\
類\\
編\\
號\\
\vspace{0.1em}
\hspace{-0.5em}:\\
\vspace{0.1em}
\rotatebox[origin=cc]{270}{\sectionef\LARGE \textbf{113-4-APP-3004-1}}\\ %旋轉
\vspace{0.1em}
O\\
D\\
O\\
O\\
P\\
L\\
M\\
在\\
協\\
同\\
設\\
計\\
上\\
的\\
應\\
用\\
以\\
鋼\\
球\\
平\\
衡\\
台\\
設\\
計\\
為\\
例\\
\vspace{1em}
一\\
\vspace{0.1em}
百\\
\vspace{0.1em}
一\\
\vspace{0.1em}
十\\
\vspace{0.1em}
四\\
\vspace{0.1em}
級\\

\end{center}
%\newpage
%\begin{landscape}  %橫式環境
%\begin{center}
%\fontsize{0.001pt}{1pt}\selectfont .
%\vspace{70mm}
%\rotatebox[origin=cc]{90}{\LARGE 【14】}\rotatebox[origin=cc]%{180}{\LARGE 1-2-APP-8765} %旋轉
%\end{center}
%\end{landscape}

\end{document}
